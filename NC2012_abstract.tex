\documentclass[a4paper,11pt]{article}

% Use \includegraphics{file} if figures are needed
\usepackage{graphicx}

\usepackage{amsmath}
\usepackage{amscd}
\usepackage{amssymb}
\usepackage{amsfonts}
\usepackage{amsthm}
\usepackage{amsfonts}
\usepackage{amsthm}

\usepackage{hyperref}
%\usepackage{url}

\usepackage{subfig}

\usepackage{aviolov_style}
\usepackage{local_style}

\oddsidemargin 5mm
\evensidemargin -5mm
\topmargin -15mm
\textwidth 160mm
\textheight 24.5cm


\begin{document}

\begin{center}
\textbf{{\Large 
Inverse Problem for Leaky
Integrate-and-Fire Neuronal Models using Spike-Times Data
}}

\medskip

\textbf{{\large The sinusoidally-driven case}}

\bigskip

\textit{Alexandre Iolov},
\textit{Andr\'e Longtin} \\
University of Ottawa  \\
Physics Department, 150 Louis Pasteur \\
Ottawa, Ontario, Canada K1N 6N5  \\
\{alongtin,aiolo040\} at uottawa dot ca \\ 
\end{center}

\section*{Abstract} We aim to estimate structural parameters (the coefficients)
of a stochastic leaky-integrate and fire neuronal model given only data for the
interspike intervals (ref.\ \cite{Meng2011,Mullowney2008a,Ditlevsen2007}).
Building on the Fortet Integral Equation method introduced for this problem in \cite{Ditlevsen2008} and the classical Fokker-Plank
equation, we extend the methods to the case of a time-dependent drift, in
particular a sinusoidal term with unknown amplitude, a situation common for
sensory neurons \cite{Chacron2000}. 

Thus the voltage trajectory, $X$, is assumed to follow the following trajectory
(in non-dimensional form):
\begin{equation}
dX_s = (\a - X_s + \g \sin( \th (s+\phi) ) ) \intd{s} + \b \intd{W_s} .
\label{eq:X_evolution_uo}
\end{equation}
until $X$ hits a threshold at which time a spike is recorded and $X$ is reset.
Here, $intd{W_s}$ is an increment from standard Brownian motion.

The main difficulty in estimating the parameters for these dynamics from
spike-data only is that the observed interspike-intervals of the inferred
stochastic process do not form a renewal process due to the phase differences in
the sinusoidal term at the beginning of each interval (at each spike). We
propose to deal with this difficulty by binning the data into several bins each
of which with approximately the same phase and then to treat each bin with the
methods developed for renewal processes (the Fortet Equation and the
Fokker-Plank equation). Both algorithms are iterative and so we also introduce a
simple and constructive initialization procedure to provide them with initial
guesses for the parameters directly and automatically from the data. The
initializer works on approximating the distribution of spike dynamics as an
advected Gaussian bell, whose mean and std. deviation can be estimated from the
quantiles of the ISI distribution.

\section*{Results to Present}
We attempt to estimate the parameters for four (4) distinct regimes, which we
call: 'Super-Threshold', 'Critical', 'Sub-Threshold' and 'Super-Sinusoidal'. The
regimes' respective parameters are in tab.\ \ref{tab:regimes}. These regimes try
to represent the following scenarios: in the 'Super-Threshold', the bias
current, $\a$ is strong enough to trigger spikes on its own and the noise and
sinusoidal currents only create jitter around the regular spiking;  in the
'Critical' regime the two deterministic currents, $\a$ and $\g \sin()$ are
barely strong enough to spike without noise; in the 'Sub-Threshold', the neuron
spikes only due to the stochastic fluctuations; while the 'Super-Sinusoidal' is
like 'Super-Threshold', but here it is the sinusoidal $\g \sin()$ term that
dominates and that is the primary driver of the spikes. 

The estimation results for 16 spike trains are shown in Fig.\
\ref{fig:comprehensive_tests}. We see that the estimation procedures are most
accurate for the 'Super-Threshold' and the 'Critical' regimes. We also find the
Fortet-equation method to be more accurate usually, but less robust, meaning it
is more likely to be way off. In contrast, the Fokker-Plank-equation method is
more robust. We also find that the proposed initializer is doing an effective
job, starting the iterative methods close to the true parameter values. 

\begin{table}[h]
\begin{center}
\begin{tabular}{l|ccc}
Regime Name & $\a$ & $\b$ & $\g$ \\
\hline
Super Threshold & 
1.5 & 0.3 & 1.0 \\ 
Critical &
0.55 & 0.5 & .55 \\ 
Sub-Threshold &
0.4 & 0.3 & .4 \\ 
Super Sinusoidal &
0.1 & 0.3 & 2  
\end{tabular}
\caption{$\abg$ parameters for the different regimes.}
\label{tab:regimes}
\end{center}
\end{table} 

\begin{figure}[h!]
\begin{center}
\subfloat[SUPER THRESHOLD]
{
\label{fig:comp_test_superT}
\includegraphics[width=0.48\textwidth]
{Figs/Estimates/superT_est_errors.png}
}
\subfloat[CRITICAL]
{
\label{fig:comp_test_critical}
\includegraphics[width=0.48\textwidth]
{Figs/Estimates/crit_est_errors.png} 
}
\\
\subfloat[SUB THRESHOLD]
{
\label{fig:comp_test_subT}
\includegraphics[width=0.48\textwidth]
{Figs/Estimates/subT_est_errors.png}
}
\subfloat[SUPER SINUSOID]
{
\label{fig:comp_test_superSin}
\includegraphics[width=0.48\textwidth]
{Figs/Estimates/superSin_est_errors.png}
}
\caption{Absolute errors of the parameter estimation routines for the 4
different spike regimes. The upper panel shows the difference
between the estimated and the real value for each parameter, e.g. $\tilde{\a} -
\a$. The lower panel shows the sum of the absolute errors, i.e.\ $|\tilde{\a} - \a|+|\tilde{\b}
- \b| +|\tilde{\g} - \g|$.
NOTE: figures have different scales.}
\label{fig:comprehensive_tests}
\end{center}
\end{figure}

\medskip\noindent\textbf{\textit{Keywords:}}
inverse first-passage times, stochastic neuron models, parameter estimation from
stopping times, Fortet integral equation, Fokker-Plank equation.

\bibliographystyle{plain}
\bibliography{library}

\end{document}
