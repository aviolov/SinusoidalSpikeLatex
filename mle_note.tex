% %This is a very basic article template. %There is just one section and two
% subsections.
\documentclass{article}

\usepackage{amsmath}
\usepackage{amscd}
\usepackage{amssymb}
\usepackage{amsfonts}
\usepackage{amsthm}
\usepackage{amsfonts}
\usepackage{amsthm}

% \usepackage{circuitikz}
\usepackage{pgf}
\usepackage{tikz}
\usetikzlibrary{arrows,snakes,backgrounds} 
% \usetikz
\usepackage{subfig}

\usepackage[super]{nth}
% \usepackage{appendix} \usepackage{listings} \usepackage{color}

\usepackage{hyperref}
% \usepackage{url}

\usepackage{cleveref}

\usepackage{aviolov_style}
\usepackage{local_style}
 
\begin{document}

\title{Maximum Likelihood Estimates for the  leaky integrate-and-fire model with
sinusoidal and stochastic forcing}

\author{Alexandre Iolov, Susanne Ditlevsen and Andr\'e Longtin }
 
\date{\today}

\maketitle

% \maketitle \tableofcontents

\vskip 20pt

\section{Maximum Likelihood Estimation Procedure}
We detail the maximum likelihood (ML) approach to estimating the parameters
$\abg$ in the sinusoidal, noisy LIF model as discussed in the main text
\cite{Iolov2013}. We assume the reader is familiar with the notions and notation
in \cite{Iolov2013}, most pertinently the material in sec.\ 3.2.

We start by recalling the relationship between the ISI survival density, $g$,
the survival distribution $\G$ and the Fokker-Planck transition distribution,
$F$. $$\G_{\phi}(t) = F^{(\phi)}(1,t)$$ and $$g_{\phi}( t)  = -\di_t
\G_{\phi}(t). $$

Given a set of $N$ observed spikes and phase angles, $(i_n,
\phi_{n-1})_{n=1}^N$, the ML estimates are obtained by considering the
log-likelihood function, $L$,
\begin{equation}
 L(\a,\b,\g) = \sum_n  \log (
g_{\phi_{n-1}}(i_n) ) = \sum_n \log \big[ -\di_t F^{\phi_{n-1}}_{\a,\b,\g}(\xth, t) \big] \Big|_{t =
i_n}. 
\label{eq:loss_function_MLE}
\tag{A1}
\end{equation}
and maximizing it,$\abgest = \argmax L.$
Theoretically, one could stop there and start the number crunching. However, for
each evaluation of $L$ we would have to solve $N$ PDEs for $F$. That might
become computationally burdensome. Instead, in keeping with the spirit of the
paper, we choose a set of $M$ representative $\phi$'s $\{\phi_m\}$, chosen as
the same phase bin midpoints as in the main text, and approximate $\G_\phi$ as
a weighted average of $\{F^{\phi_m} \}$.



In particular given $\phi \in [\phi_m, \phi_{m+1}]$, we take a simple
linear average:
$$\G_\phi(t) \approx
\frac{\phi_{m+1} - \phi}{\phi_{m+1} - \phi_{m}} \cdot F^{\phi_m}(t)
+ 
\frac{\phi - \phi_{m}}{\phi_{m+1} - \phi_{m}} \cdot F^{\phi_{m+1}}(t).
$$
The basic idea is that if $\phi \approx \phi_m$ we use $F^{\phi_m}$ and
conversely if $\phi \approx \phi_{m+1}$ we use $F^{\phi_{m+1}}$.
Note that this assures that $0 \leq \G \leq 1$.

Finally, recall that $F$ is solved numerically, using a simple finite difference
to estimate the derivative in \cref{eq:loss_function_MLE}. If a spike time $i_n$
falls between two time slices $t_k, t_{k+1}$, $i_n \in [t_k, t_{k+1})$ then we approximate $-\di_t F(\xth, i_n)$ as $$ -\di_t F(\xth, i_n)
\approx -\frac{F(\xth, t_{k+1}) - F(\xth, t_{k})}{t_{k+1} - t_{k}}. $$


Putting it all together, the estimates are obtained as the maximizers of 
\begin{align}
\tag{A2}
\label{eq:loss_function_ML}
L(\a,\b,\g) = \sum_n  \log \Bigg(
&-\frac{\phi_{m+1} - \phi}{\phi_{m+1} - \phi_{m}} \cdot 
\frac{F_{\a,\b,\g}^{\phi_m}(\xth, t_{k+1}) - F_{\a,\b,\g}^{\phi_m}(\xth,
t_{k})}{t_{k+1} - t_{k}}
\\&-
\frac{\phi - \phi_{m}}{\phi_{m+1} - \phi_{m}} \cdot 
\frac{F_{\a,\b,\g}^{\phi_{m+1}}(\xth, t_{k+1}) - F_{\a,\b,\g}^{\phi_{m+1}}(\xth,
t_{k})}{t_{k+1} - t_{k}} 
 \Bigg)
 \Bigg|_{\substack{i_n \in [t_k, t_{k+1}] \\
 			 \phi_{n-1} \in [\phi_m, \phi_{m+1}]}} .
 			 \notag
\end{align} 
after numerically solving the PDE for $F$, eq. 9 in \cite{Iolov2013}. 

To maximize $L$ we again use the Nelder-Mead optimizer from the SciPy library,
\cite{scipy}. We also tried the gradient-based optimizers in SciPy (BFGS,
Truncated Newton, Sequential Least Squares), but they turned out slower and
provided less accurate estimates.
  
\section{ML Estimates}
We label estimates based on maximizing \cref{eq:loss_function_ML} as ML
estimates, while estimates obtained by minimizing eq.\ 17 in the main text, \cite{Iolov2013},
are called FP estimates. Of course, both of them rely on solving the
Focker-Planck equation, but we do this in order to be consistent with the main text
where estimates based on eq. 17 are labeled with 'FP'. We summarize the two
estimators in \cref{tab:estimators}. \begin{table} \begin{tabular}{ccp{4cm}}
Name & Loss function $L$ & Short description
\\
\hline FP &$\sum_{m} N_m \Big\{ \sup_{t} \left| \Gest_{\phi_m}(t) -
F^{\phi_m}_{\a,\b,\g}(\xth,
t) \right| \Big \}$ & Minimize the $sup$ of the  differences between a
numerically calculated survivor distribution and the empirically observed
survivor function
\\
ML & $-\sum_n  \log \Big( g_\phi(i_n)
 \Big)$
& Maximize the sum of the log-likelihoods of the observed ISIs
 		\\
\hline  \end{tabular} \caption{The two estimators we compare: the ML estimator
and  the FP estimator, which has already been discussed at length in the main text,
\cite{Iolov2013}} \label{tab:estimators} \end{table}


To compare the FP vs.\ ML estimators, we draw cross-plots between
the two in the same style as figs. 12,13 in the main text,
\cite{Iolov2013}. As always, we use 100 spike trains to obtain 100
estimates for $\abg$ using each of the two methods. In all simulations, we set
$\th = 1$.

We use the same regimes, super-threshold, super-sinusoidal, critical,
sub-threshold, with the same parameters as in the main text,

We start by using $N=100$ spikes per spike train - see
\cref{fig:MLCompare_N100}. We use 16 bins for the ML estimate and 8 for the FP.
The differences between the two methods appear minor. Neither method seems to be
consistently better than the other. In some cases, the ML estimators are
noticeably worse - especially for the super-sinusoidal regime, where the ML
method shows a bimodal distribution of the $\a$ parameter. In other cases, the
ML estimator has a smaller variance - see $\g$ in the critical regime and maybe
$\a$ in the supra-threshold regime, but that is already estimated quite
accurately.


Next, we try using the larger sample size, i.e.\ $N=1000$ spikes per spike train
- see \cref{fig:MLCompare_N1000}. Again the differences are minor and similar to
what was seen in the smaller sample size, $N=100$. Most notably the ML is now
better at estimating $\a$ in the super-sinusoidal regime, except for one rare
case.


At this point, we would conclude that the performance of the ML and FP
methods is comparable. However the ML method has the potential advantage that
one can use much smaller bins (more $\phi_m$s) no matter how big or small the
data size. This is because the FP method relies on there being a sufficient
amount of spikes in each bin in order to approximate the survivor distribution,
while the ML method does not have this requirement.

So as a final comparison we redo the estimation for $N=100$, but now
we use $M=32$ bins for the ML method while still using only $M=8$ bins for
FP. The results are in \cref{fig:MLCompare_N100_32bins}. It is evident
that the differences between using \cref{fig:MLCompare_N100} using 16 bins for
the ML method and \cref{fig:MLCompare_N100_32bins} using 32 bins are not
significant.

As a final note, we mention that the computing times for both methods are
roughly the same, modulo that one uses the same number of bins.

As such we conclude that the FP estimates and the ML estimates behave comparably.
 
\begin{figure}[htp]
\begin{center}
  \includegraphics[width=1\textwidth]{Figs/Estimates/MLE_N100_cross_compare_joint.pdf}
  \caption{Comparison between the FP vs.\ ML
  estimators in the four regimes, using $N=100$ spikes per sample. We used 8
  bins for the FP and 16 bins for the ML. In all simulations $\th = 1$.}
  \label{fig:MLCompare_N100}
\end{center}
\end{figure}

\begin{figure}[htp]
\begin{center}
  \includegraphics[width=1\textwidth]{Figs/Estimates/MLE_N1000_cross_compare_joint.pdf}
  \caption[labelInTOC]{Comparison between the FP vs.\ ML
  estimators in the four regimes, using $N=1000$ spikes per sample. We used 8
  bins for the FP and 16 bins for the ML. In all simulations $\th = 1$.}
  \label{fig:MLCompare_N1000}
\end{center}
\end{figure}
% 
% \begin{figure}[htp]
% \begin{center}
%   \includegraphics[width=1\textwidth]{Figs/Estimates/ML_N100_bfgs_cross_compare_joint.pdf}
%   \caption[labelInTOC]{BFGS, N=100} 
%   \label{fig:MLCompare_N100_BFGS} 
% \end{center}
% \end{figure}

% \begin{figure}[htp]
% \begin{center}
%   \includegraphics[width=1\textwidth]{Figs/Estimates/ML_N100_tnc_cross_compare_joint.pdf}
%   \caption[labelInTOC]{TNC, N=100} 
%   \label{fig:MLCompare_N100_TNC}
% \end{center}
% \end{figure}

\begin{figure}[htp]
\begin{center}
  \includegraphics[width=1\textwidth]{Figs/Estimates/MLE_N100_32bins_cross_compare_joint.pdf}
  \caption{Same as \cref{fig:MLCompare_N100}, but using 32
  bins for ML instead of 16. $N=100$. In all simulations $\th = 1$.}
  \label{fig:MLCompare_N100_32bins}
\end{center}
\end{figure}



\bibliographystyle{plain}
\bibliography{library,local}

\end{document}