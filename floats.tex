\begin{figure}[ht] 
\begin{center} 
\includegraphics[width=0.95\textwidth]{Figs/Trajectories/path_T=24_combined.pdf} 
\end{center} 
\caption{Example trajectories from \cref{eq:X_evolution_uo} for 
the four different parameter regimes using the parameter values given in 
\cref{tab:regimes}. A) supra-threshold, B) super-sinusoidal, C) critical, D) 
sub-threshold. In the supra-threshold regime 
spikes occur regularly and often; in the super-sinusoidal regime 
spikes cluster near the peak of the sine wave; in the critical regime 
they occur less often; and in the sub-threshold regime, spikes occur 
rarely. For all regimes, $\th = 1$.}  
\label{fig:trajectory_examples}     
\end{figure} 
\begin{figure}[ht] 
\begin{center} 
\includegraphics[width=0.99\textwidth]{Figs/FP/regimes_pdf_combined.pdf} 
\caption{The four different parameter regimes using the parameter values given in 
\cref{tab:regimes}. Illustrated are the 
probability density functions, $g_{\phi_m}(t)$, for representative $\phi_m = 
2\pi/\th \times \{0,0.25,0.5,0.75\}$.  
Varying $\phi_m$ has, 
 for the most part, the effect of shifting the curves laterally,  
 while varying $\abg$ changes their characteristic form. For all regimes, $\th 
 = 1$. 
 A) supra-threshold, B) super-sinusoidal, C) critical, D) sub-threshold} 
\label{fig:4regimes_illustrated_PDF}   
\end{center}       
\end{figure}              
\begin{figure}[ht]     
\begin{center}  
\includegraphics[width=0.99\textwidth]{Figs/FP/regimes_sdf_combined.pdf} 
\caption{The four different parameter regimes using the parameter values given in 
\cref{tab:regimes}. Illustrated are the survivor 
distribution functions, $\G_{\phi_m}(t)$, for representative $\phi_m = 2\pi/\th 
\times \{0,0.25,0.5,0.75\}$. Varying $\phi_m$ has, for the most part, the effect 
of shifting the curves laterally, while varying $\abg$ changes their 
characteristic form. 
A) supra-threshold, B) super-sinusoidal, C) critical, D) sub-threshold.} 
\label{fig:4regimes_illustrated_SDF}     
\end{center} 
\end{figure}      
\begin{figure}[h] 
\begin{center} 
\includegraphics[width=0.95\textwidth]{Figs/FP/SDF3D_combined.pdf}  
\caption{Example solution to \cref{eq:FP_pde_OU_absorbBC_CDF} for 
$(\a,\b,\g) = (0.5, 0.3, 0.5\sqrt{2})$; $\th= 1, \phi = \pi / 2$. 
In A,B,C, we show the full solution in space-time $F(x,t)$. In (d) we show 
the time solution at the upper boundary, $F(1,t)$.}  
\label{fig:FP_pde_OU_absorbBC_CDF}  
\end{center} 
\end{figure} 
\begin{figure}[ht] 
\begin{center} 
\includegraphics[width=0.9\textwidth]{Figs/Bins/Example_composite.pdf} 
  \end{center} 
\caption{ The raw $(\In, \phi_n)$ pairs (left) are binned into a set of $M$ bins 
with a representative $\phi_m$ (right) and the ISIs within each bin are treated 
as a renewal process. In this illustration, $M=8$, $\th = 
1$ while the parameters $\abg$ are taken from the 
supra-threshold regime. } 
\label{fig:binning_visualized}  
\end{figure} 
\begin{figure}[h] 
\begin{center} 
\includegraphics[width=0.99\textwidth]{Figs/FP/EffectOfM_Referees.pdf} 
\caption{Effect of $M$, the number of bins, on the approximate survival 
distribution. The full-drawn blue curve is the true survivor distribution 
given in \cref{eq:FP_pde_OU_absorbBC_CDF}, the red points are the approximation 
given in \cref{eq:SDF_estimate_per_bin}.   
In the figures, the least populous (above) and most populous (below) bin for 
each $M$ is shown. The width of the bins is $\dphi = {2\pi}/{(\th M)}$. 
We have used A,E) $M=5$, B,F) $M=10$, C,G) $M=20$, 
D,I) $M=40$. As $M$ increases, the approximation of using the survival 
distribution using $\phi_m$ instead of $\phi_n$ improves since $\phi_m(n) \ra 
\phi_n$ as $M \ra \infty$. The data is generated using parameter 
values from the super-sinusoidal regime and $N=1000$. For this 
particular data set the largest generated ISI was 6.55 time units.} 
\label{fig:effect_of_M} 
\end{center} 
\end{figure} 
\begin{figure}[htp] 
\begin{center} 
\includegraphics[width=.99\textwidth]{Figs/FP/sdf_init_vs_exact.pdf} 
\caption[labelInTOC]{The blue curves are the numerically obtained survivor 
  distributions $\G_\phi$ for the exact parameters in the four regimes (as in 
  \cref{tab:regimes}) and $\th=1$. The red curves are obtained in the following 
  manner: Simulations using the true parameters were used to generate sample spikes. 
  Using these samples, the initializer algorithm was used to generate estimates 
  for $\a,\b,\g$. Using these estimates, the bell curve discussed in sec.\ 3.4 
  was formed and evolved in time.  
  Thus, the red curve drawn in the figures measures the area under 
  this bell that is to the left of the threshold at time $t$.  
 A) supra-threshold, B) super-sinusoidal, C) critical, D) sub-threshold.} 
  \label{fig:sdf_real_vs_init_estimated} 
\end{center} 
\end{figure} 
\begin{figure}[p] 
\begin{center} 
\includegraphics[width=0.99\textwidth] 
{Figs/Estimates/FP_vs_Fortet_100x100_x1000superT_est_rel_errors_joint.pdf} 
\caption{Boxplots of parameter estimates for the 
supra-threshold regime. The upper plots (A,B,C) show estimates using $N=100$ sample spikes per 
estimation, while the lower plots (D,E,F) use $N=1000$. The dashed line 
indicates the true parameter value, while the red line inside the boxes 
indicates the median of the estimates. 
\\ 
The boxes contain the central 50\% of the estimates. The bars indicate 
the range of the estimates, except for outliers given by the points 
outside the bars, and defined to be more than 1.5 times the 
interquantile range (the height of the box) from the box.} 
\label{fig:comprehensive_test_SuperT_relerrors} 
\end{center} 
\end{figure} 
\begin{figure}[p] 
\begin{center} 
\includegraphics[width=0.99\textwidth]{Figs/Estimates/FP_vs_Fortet_100x100_x1000superSin_est_rel_errors_joint.pdf} 
\caption{Boxplots of parameter estimates for the 
super-sinusoidal regime. The upper plots (A,B,C) show estimates using $N=100$ 
sample spikes per estimation, while the lower plots (D,E,F) use $N=1000$. The dashed line 
indicates the true parameter value, while the red line inside the boxes 
indicates the median of the estimates.\\ 
The boxes contain the central 50\% of the estimates. The bars indicate 
the range of the estimates, except for outliers given by the points 
outside the bars, and defined to be more than 1.5 times the 
interquantile range (the height of the box) from the box.} 
\label{fig:comprehensive_test_SuperSin_relerrors} 
\end{center} 
\end{figure} 
\begin{figure}[p] 
\begin{center} 
\includegraphics[width=0.99\textwidth]{Figs/Estimates/FP_vs_Fortet_100x100_x1000crit_est_rel_errors_joint.pdf} 
\caption{Boxplots of parameter estimates for the 
critical regime. 
The upper plots (A,B,C) show estimates using $N=100$ sample spikes per 
estimation, while the lower plots (D,E,F) use $N=1000$. The dashed line 
indicates the true parameter value, while the red line inside the boxes 
indicates the median of the estimates. 
\\ 
The boxes contain the central 50\% of the estimates. The bars indicate 
the range of the estimates, except for outliers given by the points 
outside the bars, and defined to be more than 1.5 times the 
interquantile range (the height of the box) from the box.}   
\label{fig:comprehensive_test_crit_relerrors} 
\end{center}     
\end{figure}  
\begin{figure}[p]   
\begin{center} 
\includegraphics[width=0.99\textwidth]{Figs/Estimates/FP_vs_Fortet_100x100_x1000subT_est_rel_errors_joint.pdf} 
\caption{Boxplots of parameter estimates for the  
sub-threshold regime. 
The upper plots (A,B,C) show estimates using $N=100$ sample spikes per 
estimation, while the lower plots (D,E,F) use $N=1000$. The dashed line 
indicates the true parameter value, while the red line inside the boxes 
indicates the median of the estimates. 
\\ 
The boxes contain the central 50\% of the estimates. The bars indicate 
the range of the estimates, except for outliers given by the points 
outside the bars, and defined to be more than 1.5 times the 
interquantile range (the height of the box) from the box.} 
\label{fig:comprehensive_test_SubT_relerrors} 
\end{center} 
\end{figure} 
\begin{figure}[htp] 
\begin{center} 
\includegraphics[width=0.99\textwidth] 
{Figs/Estimates/FP_vs_Fortet_100x100_cross_compare_joint.pdf} 
\caption{Estimates based on samples of $N = 100$ spikes obtained from the 
Fokker-Planck algorithm against the Fortet algorithm for the four different 
parameter regimes, with parameter values given in table 
\cref{tab:regimes}, fixing $\th=1$. Each row corresponds to one regime 
and one set of simulations. Each column corresponds to a parameter, 
with the specific value indicated above each plot.   
A,B,C) Supra-threshold; D,E,F) Super-sinusoidal; G,H,I)  
Critical; J,K,L) Sub-threshold.} 
\label{fig:comprehensive_tests_cross_comparison} 
\end{center} 
\end{figure} 
\begin{figure}[htp] 
\begin{center} 
\includegraphics[width=0.99\textwidth]     
{Figs/Estimates/FP_vs_Fortet_100x1000_cross_compare_joint.pdf} 
\caption{Estimates based on samples of $N = 1000$ spikes obtained from the 
Fokker-Planck algorithm against the Fortet algorithm for the four different 
parameter regimes, with parameter values given in table 
\cref{tab:regimes}, fixing $\th=1$. Each row corresponds to one regime 
and one set of simulations. Each column corresponds to a parameter, 
with the specific value indicated above each plot.   
A,B,C) Supra-threshold; D,E,F) Super-sinusoidal; G,H,I)  
Critical; J,K,L) Sub-threshold.} 
\label{fig:comprehensive_tests_cross_comparison} 
\end{center} 
\end{figure} 
\begin{figure}[htp] 
\begin{center} 
\includegraphics[width=0.99\textwidth]   
{Figs/Estimates/thetas_100x1000thetas_est_rel_errors.pdf} 
\caption{Boxplots of parameter estimates for varying $\th$ across $[1, 5, 10, 20]$ while holding $\g / 
\sqrt{1+\th^2}$ constant as to keep the parameters in the critical 
regime. 
A-C) $\th=1$,  D-F) $\th=5$,         
G-I) $\th=10$, J-L) $\th=20$. 
\\ 
The boxes contain the central 50\% of the estimates. The bars indicate 
the range of the estimates, except for outliers given by the points 
outside the bars, and defined to be more than 1.5 times the 
interquantile range (the height of the box) from the box.}   
\label{fig:comprehensive_test_thetas_relerrors}     
\end{center} 
\end{figure}    
\begin{figure}[htp] 
\begin{center} 
\includegraphics[width=0.99\textwidth] 
{Figs/Estimates/FP_vs_Fortet_thetas_cross_compare_joint.pdf}  
\caption{Estimates based on samples of $N = 1000$ spikes obtained from the 
Fokker-Planck algorithm against the Fortet algorithm for a parameter set in the 
critical regime, while varying $\th$ across $[1, 5, 10, 20]$ and holding  
$\g / \sqrt{1+\th^2}$ and $\a$ constant. 
A,B,C) $\th=1$; D,E,F) $\th=5$; G,H,I) 
$\th=10$; J,K,L) $\th=20$. }  
\label{fig:comprehensive_test_thetas_cross_compare} 
\end{center} 
\end{figure} 
\begin{figure}[htp] 
\begin{center} 
\includegraphics[width=0.99\textwidth]   
{Figs/Estimates/thetavariation_100x1000_alphagamma_compare_joint.pdf} 
\caption{Comparison of $\aest$ vs. $\gest$ parameter estimates while varying 
$\th$ across $[1, 5, 10, 20]$,  holding $\g / \sqrt{1+\th^2}$ constant as to 
keep the parameters in the critical regime. 
A,B,C) $\th=1$; D,E,F) $\th=5$; G,H,I) 
$\th=10$; J,K,L) $\th=20$.    
} 
\label{fig:comprehensive_test_thetas_alpha_vs_gamma} 
\end{center} 
\end{figure} 
\clearpage
\begin{table}[ht]

\begin{center}
\begin{tabular}{l|ccc}
Regime Name & $\a$ & $\b$ & $\g$ \\ \hline
Supra-threshold&1.40&0.30&0.14 \\
Super-sinusoidal&0.10&0.30&1.98 \\
Critical&0.50&0.30&0.71 \\
Sub-threshold&0.40&0.30&0.57 \\
\end{tabular}
\caption{Example of $\abg$ parameter values for the different regimes, given $\th = 1$.}
\label{tab:regimes}
\end{center}
\end{table}
\begin{table}

\begin{center}
{\begin{tabular}{|c|ccc|} 
Parameter
& Initializer
& Fokker-Planck
& Fortet
\\ \hline
\multicolumn{4}{|c|}{Supra-threshold regime} \\[1mm]
$\alpha=1.40$
& $1.43 : [1.29, 1.56]$
& $1.34 : [1.24, 1.43]$
& $1.41 : [1.33, 1.49]$
\\
$\beta=0.30$
& $0.17 : [0.10, 0.24]$
& $0.29 : [0.21, 0.39]$
& $0.29 : [0.22, 0.36]$
\\
$\gamma=0.14$
& $0.16 : [0.02, 0.33]$
& $0.12 : [0.02, 0.23]$
& $0.12 : [0.01, 0.24]$
\\
\hline \hline
\multicolumn{4}{|c|}{Super-sinusoidal regime} \\[1mm]
$\alpha=0.10$
& $0.92 : [0.83, 1.01]$
& $0.28 : [0.02, 0.59]$
& $0.24 : [-0.22, 0.42]$
\\
$\beta=0.30$
& $0.15 : [0.10, 0.25]$
& $0.31 : [0.14, 0.53]$
& $0.32 : [0.14, 0.46]$
\\
$\gamma=1.98$
& $1.35 : [1.13, 1.57]$
& $1.67 : [1.33, 2.05]$
& $1.77 : [1.44, 2.38]$
\\
\hline \hline
\multicolumn{4}{|c|}{Critical regime} \\[1mm]
$\alpha=0.50$
& $0.72 : [0.66, 0.80]$
& $0.57 : [0.32, 0.73]$
& $0.57 : [0.36, 0.73]$
\\
$\beta=0.30$
& $0.19 : [0.10, 0.26]$
& $0.27 : [0.17, 0.40]$
& $0.25 : [0.15, 0.40]$
\\
$\gamma=0.71$
& $0.57 : [0.44, 0.73]$
& $0.55 : [0.30, 0.83]$
& $0.62 : [0.38, 0.93]$
\\
\hline \hline
\multicolumn{4}{|c|}{Sub-threshold regime} \\[1mm]
$\alpha=0.40$
& $0.62 : [0.57, 0.67]$
& $0.63 : [0.33, 0.84]$
& $0.58 : [0.03, 1.00]$
\\
$\beta=0.30$
& $0.17 : [0.10, 0.29]$
& $0.20 : [0.10, 0.37]$
& $0.19 : [0.00, 0.41]$
\\
$\gamma=0.57$
& $0.32 : [0.00, 0.53]$
& $0.29 : [0.00, 0.62]$
& $0.46 : [0.00, 1.19]$
\\
\hline
 \end{tabular}}\\
\end{center}
\caption{Averages and empirical 95\% confidence intervals of the estimates for
$N=100$ spikes per train.}
\label{tab:est_quantiles_100}
\end{table}
\begin{table}

\begin{center}
{\begin{tabular}{|c|ccc|} 
Parameter
& Initializer
& Fokker-Planck
& Fortet
\\ \hline
\multicolumn{4}{|c|}{Supra-threshold regime} \\[1mm]
$\alpha=1.40$
& $1.44 : [1.40, 1.50]$
& $1.36 : [1.33, 1.40]$
& $1.40 : [1.37, 1.42]$
\\
$\beta=0.30$
& $0.25 : [0.22, 0.28]$
& $0.29 : [0.26, 0.32]$
& $0.30 : [0.27, 0.32]$
\\
$\gamma=0.14$
& $0.14 : [0.10, 0.19]$
& $0.14 : [0.10, 0.17]$
& $0.14 : [0.10, 0.18]$
\\
\hline \hline
\multicolumn{4}{|c|}{Super-sinusoidal regime} \\[1mm]
$\alpha=0.10$
& $0.90 : [0.85, 0.92]$
& $0.11 : [0.03, 0.29]$
& $0.10 : [0.03, 0.16]$
\\
$\beta=0.30$
& $0.18 : [0.14, 0.23]$
& $0.30 : [0.21, 0.34]$
& $0.31 : [0.22, 0.34]$
\\
$\gamma=1.98$
& $1.26 : [1.16, 1.34]$
& $1.92 : [1.49, 2.05]$
& $1.96 : [1.86, 2.07]$
\\
\hline \hline
\multicolumn{4}{|c|}{Critical regime} \\[1mm]
$\alpha=0.50$
& $0.73 : [0.70, 0.75]$
& $0.51 : [0.43, 0.63]$
& $0.53 : [0.45, 0.64]$
\\
$\beta=0.30$
& $0.20 : [0.17, 0.24]$
& $0.29 : [0.24, 0.32]$
& $0.28 : [0.19, 0.33]$
\\
$\gamma=0.71$
& $0.54 : [0.44, 0.61]$
& $0.66 : [0.52, 0.76]$
& $0.67 : [0.54, 0.77]$
\\
\hline \hline
\multicolumn{4}{|c|}{Sub-threshold regime} \\[1mm]
$\alpha=0.40$
& $0.62 : [0.55, 0.65]$
& $0.57 : [0.45, 0.66]$
& $0.56 : [0.26, 0.71]$
\\
$\beta=0.30$
& $0.20 : [0.17, 0.26]$
& $0.22 : [0.18, 0.29]$
& $0.21 : [0.13, 0.35]$
\\
$\gamma=0.57$
& $0.36 : [0.18, 0.44]$
& $0.36 : [0.25, 0.50]$
& $0.43 : [0.28, 0.72]$
\\
\hline
 \end{tabular}}\\
\end{center}
\caption{Averages and empirical 95\% confidence intervals of the estimates for $N=1000$
spikes per train. }
\label{tab:est_quantiles_1000}
\end{table}
\begin{table}

\begin{center}
\subfloat[N=100]{
\begin{tabular}{c|cc|}
Regime & Fortet & Fokker-Planck \\
\hline
Sub-threshold
& 1.29 $\pm$ 0.72
& 0.52 $\pm$ 0.21
\\
Supra-threshold
& 0.83 $\pm$ 0.28
& 0.18 $\pm$ 0.20
\\
Critical
& 0.94 $\pm$ 0.42
& 0.36 $\pm$ 0.16
\\
Super-sinusoidal
& 1.36 $\pm$ 0.46
& 0.43 $\pm$ 0.17
\\
\end{tabular}
}
\subfloat[N=1000]{
\begin{tabular}{|c|cc}
Regime & Fortet & Fokker-Planck \\
\hline
Sub-threshold
& 9.68 $\pm$ 4.98
& 1.69 $\pm$ 0.91
\\
Supra-threshold
& 3.90 $\pm$ 1.05
& 0.21 $\pm$ 0.06
\\
Critical
& 10.03 $\pm$ 2.88
& 1.28 $\pm$ 0.41
\\
Super-sinusoidal
& 10.13 $\pm$ 2.24
& 1.06 $\pm$ 0.33
\\
\end{tabular}
}
\end{center}
\caption{Average times $\pm$ standard deviations in seconds for the algorithm
in various regimes. Left: $N = 100$ spikes; right: $N = 1000$ spikes.}
\label{tab:walltimes}
\end{table} 
\begin{table}[htp]

\begin{center}
{\begin{tabular}{|c|ccc|} 
Parameter
& Initializer
& Fokker-Planck
& Fortet
\\ 
\hline \hline
\multicolumn{4}{|c|}{$\Omega=1$} \\[1mm]
$\alpha=0.50$
& $0.73 : [0.69, 0.75]$
& $0.52 : [0.45, 0.61]$
& $0.52 : [0.44, 0.62]$
\\
$\beta=0.30$
& $0.20 : [0.17, 0.25]$
& $0.29 : [0.24, 0.33]$
& $0.29 : [0.22, 0.34]$
\\
$\gamma=0.71$
& $0.54 : [0.44, 0.62]$
& $0.64 : [0.53, 0.75]$
& $0.68 : [0.55, 0.81]$
\\
\hline \hline
\multicolumn{4}{|c|}{$\Omega=5$} \\[1mm]
$\alpha=0.50$
& $0.88 : [0.76, 0.99]$
& $0.78 : [0.61, 0.89]$
& $0.64 : [0.39, 0.99]$
\\
$\beta=0.30$
& $0.24 : [0.17, 0.31]$
& $0.26 : [0.20, 0.34]$
& $0.27 : [0.12, 0.34]$
\\
$\gamma=2.55$
& $0.85 : [0.00, 1.65]$
& $0.92 : [0.00, 1.68]$
& $1.86 : [0.00, 3.10]$
\\
\hline \hline
\multicolumn{4}{|c|}{$\Omega=10$} \\[1mm]
$\alpha=0.50$
& $0.90 : [0.78, 0.99]$
& $0.71 : [0.52, 0.88]$
& $0.58 : [0.37, 0.86]$
\\
$\beta=0.30$
& $0.25 : [0.18, 0.33]$
& $0.26 : [0.20, 0.35]$
& $0.28 : [0.23, 0.32]$
\\
$\gamma=5.02$
& $2.82 : [0.92, 4.38]$
& $2.72 : [0.95, 3.88]$
& $4.32 : [1.20, 6.49]$
\\
\hline \hline
\multicolumn{4}{|c|}{$\Omega=20$} \\[1mm]
$\alpha=0.50$
& $0.93 : [0.76, 1.02]$
& $0.75 : [0.50, 0.92]$
& $0.62 : [0.31, 0.97]$
\\
$\beta=0.30$
& $0.27 : [0.20, 0.33]$
& $0.29 : [0.20, 0.43]$
& $0.29 : [0.25, 0.33]$
\\
$\gamma=10.01$
& $5.35 : [0.00, 12.29]$
& $3.98 : [0.00, 6.83]$
& $7.48 : [0.00, 13.96]$
\\
\hline
\end{tabular}}\\
\end{center}
\caption{Averages and empirical 95\% confidence intervals of estimates for $N=1000$
spikes per train in the critical regime for varying $\th$ across
[1,5,10,20]. Note that the upper subtable corresponds to the third
subtable in \cref{tab:est_quantiles_1000}. Numbers differ slightly due to statistical
fluctuations in the simulations. }
\label{tab:thetas_est_quantiles_1000}
\end{table}
